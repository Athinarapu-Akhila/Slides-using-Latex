\documentclass[landscape]{slides}
\usepackage{xcolor}
\pagecolor{lime}
\begin{document}
\begin{slide}
\begin{center}
\textcolor{violet}{\huge \textbf{week-7(Slides in latex)}}\\
\end{center}
\begin{center}
\textcolor{purple}{\tiny IT WORKSHOP LAB}\\
\textcolor{teal}{\tiny PRESENTATION}\\
\textcolor{blue}{\tiny Women Empowerment}\\
\end{center}
\begin{flushleft}
\scriptsize Name:A.Akhila\\
\scriptsize Roll-No.:20255A0507\\
\scriptsize class:CSE-B\\
\end{flushleft}
\end{slide}
\begin{slide}
\fcolorbox{red}{pink}{\large Women Empowerment}
\begin{itemize}
\item \tiny Empowerment is an active,multi-dimensional process which enables women to realize their full identity and powers in all spheres of life".
\item \tiny Empowerment can be defined in many ways,however, when talking about women empowerment,empowerment means accepting and allowing womens who are on the outside of the decision making process into it.
\item \tiny Peoples are empowered if they have an access to opportunities without any limitations and restrictions such as in education,profession and in their way of life.
\item \tiny Empowermwnt includes the action of raising the status of women through education,raising awareness,literacy,and training and also give training related to defence ourself.
\end{itemize}
\end{slide}
\begin{slide}
\underline{\large Types of Women Empowerment}
\begin{itemize}
\item \tiny Social Empowerment
\item \tiny Educational Empowerment
\item \tiny Economic Empowerment
\item \tiny Political Empowerment
\item \tiny Psychological Empowerment
\end{itemize}
\end{slide}
\begin{slide}
\fcolorbox{red}{blue}{\large WAYS TO ACHIEVE...}
\begin{itemize}
\item \tiny Self help groups
\item \tiny Aangan badis
\item \tiny Govt schemes
\item \tiny Micro finance
\item \tiny Self Empowerment
\end{itemize}
\end{slide}
\begin{slide}
\textcolor{violet}{\large SIX "S" FOR WOMEN EMPOWERMENT}
\begin{itemize}
\item[$\ast$]  Shiksha=Education
\item[$\ast$]  Swasthya=health
\item[$\ast$]  Swavlamban=self Reliance
\item[$\ast$]  Samajik Nyay=Justice
\item[$\ast$]  Samvedan=Sensitivity
\item[$\ast$]  Samta=Equality
\end{itemize}
\end{slide}
\end{document}
